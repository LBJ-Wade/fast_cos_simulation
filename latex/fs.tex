\documentclass[a4paper]{article}
\usepackage{amsmath,amssymb}
\usepackage{bm}
\usepackage{graphicx}

\begin{document}

\section{Background cosmology}

We only consider flat $\Lambda$CDM Universe, with matter density
$\Omega_m$ and dark energy density $\Omega_\Lambda = 1 - \Omega_m$ in units of the critical density,
\begin{equation}
  \rho_\mathrm{crit}(a) = \frac{3H(a)}{8\pi G}
\end{equation}
where Hubble parameter is,
\begin{equation}
  H(a) = H_0 \left( \Omega_m a^{-3} + \Omega_\Lambda \right).
\end{equation}

\section{Equation of motion}
Let $\bm{x}(t)$ be a comoving position of a particle at time $t$, $m$
the particle mass, and $\phi(\bm{x}, t)$ the peculiar gravitational
potential field. The Lagrangian of a particle is,
\begin{equation}
  L= \frac{1}{2} m (a\dot{x})^2 - m \phi(\bm{x}, t).
\end{equation}
and, Euler-Lagrange equation gives,
\begin{equation}
  \label{eq:motion}
  \frac{d}{dt}(a^2 \dot{\bm{x}}) + \bm{\nabla} \phi(\bm{x}, t) = 0,
\end{equation}
or,
\begin{align}
  \bm{v} &= a^2 \dot{\bm{x}},\\
  \dot{\bm{v}} &= - \bm{\nabla} \phi(\bm{x}, t)
\end{align}
where $\bm{v}$ is the \textit{canonical velocity}. 


The gravitational potential satisfies the Poisson equation,
\begin{equation}
  \left( \frac{1}{a} \nabla \right)^2 \phi(\bm{x}, t)
    = 4\pi G \left[ \rho(\bm{x}, t) - \bar{\rho}(t) \right],
\end{equation}
where $\rho$ is the matter density and $\bar{\rho}$ is the mean matter
density. Mean matter density is subtracted because it contributes to the background cosmology. Note that $\bm{\nabla}/a$ is the derivative in physical
corrdinate. This can be rewritten using the critical density,
\begin{equation}
  \label{eq:poisson2}
  \nabla^2 \phi(\bm{x}, t) =
  \frac{3}{2} H_0^2 \Omega_{m,0} \, a^{-1} \delta(\bm{x}, t)
\end{equation}
using,
\begin{align}
  \delta(\bm{x}, t) &:= \rho(\bm{x}, t)/\bar{\rho}(t) - 1 \\
  \bar{\rho}(t)     &= \Omega_{m,0} \rho_{\mathrm{crit},0} a^{-3}
\end{align}

\section{Variables in the simulation}
We use the comoving coordinate for particle positions and canonical
velocity for particles. The internal units are $h^{-1}
\mathrm{Mpc}$ for position and $100 \,\mathrm{km/s}$ for velocity, 
\begin{align}
  &\texttt{x} \leftarrow \bm{x},\\
  &\texttt{v} \leftarrow \bm{v} = a^2 \dot{\bm{x}}.
\end{align}
$H_0 = 1$ in this unit.
%
The conversion from internal unit to peculiar velocity
$\bm{v}_\mathrm{pec} = a \dot{\bm{x}}$ in $\mathrm{km}/\mathrm{s}$
requires dividing by $a$ to convert canonical velocity to peculiar
velocity and multiply by $100
\mathrm{km}$.
\begin{equation}
  \bm{v}_\mathrm{pec} = \frac{100 \,\mathrm{[km/s]}}{a} \texttt{v}
\end{equation}

The force variable, $\texttt{f}$, only contains the solution of Poisson
equation without any coefficients,
\begin{equation}
  \texttt{f} := \bm{\nabla} \nabla^{-2} \delta(\bm{x}).
\end{equation}
The acceleration is,
\begin{equation}
  \dot{\texttt{v}} = -\frac{3}{2} H_0^2 \Omega_{m,0} a^{-1} \texttt{f}.
\end{equation}

\section{Lagrangian pertubation theory}
Lagrangian variables are labelled by the initial comoving coordinate $\bm{q} = \lim_{t \rightarrow 0} x$,
\begin{equation}
  \bm{x} = \bm{q} + \bm{\Psi}(\bm{q})
\end{equation}
The mass coservation gives the expression for the density,
\begin{equation}
  d^3 q = (1 + \delta(\bm{x})) d^3 x.
\end{equation}

\subsection{Linear perturbation}

\begin{align}
  1 + \delta
    &= \left| \frac{\partial \bm{x}}{\partial \bm{q}} \right|^{-1}
     = \left|
         \begin{array}{ccc}
           1 + \partial \Psi_1/\partial q_1 & \cdot & \cdot  \\
           \cdot &  1 + \partial \Psi_2/\partial q_2 & \cdot \\
          \cdot &  \cdot & 1 + \partial \Psi_3/\partial q_3 \\
         \end{array}
       \right|^{-1}\\
       &\approx (1 + \bm{\nabla}_q \cdot \bm{\Psi})^{-1}
       \approx (1 - \bm{\nabla}_q \cdot \bm{\Psi}),
\end{align}
where the dots in the matrix are first order terms $\partial
\Psi_i/\partial q_j$.
\begin{equation}
  \delta(\bm{x}, t) = - \bm{\nabla}_q \cdot \bm{\Psi}.
\end{equation}
The divergence of the equation of motion (\ref{eq:motion}) and the
Poisson equation (\ref{eq:poisson2}) give,
\begin{equation}
  \frac{d}{dt} \left( a^2 \bm{\nabla}\cdot\bm{\Psi} \right)
  = -\frac{3}{2} H_0^2 \Omega_{m,0} a^{-1} \bm{\nabla}\cdot\bm{\Psi}.
\end{equation}
where $\bm{\nabla}_q = \bm{\nabla}$ in leading order was used.
The solution is
\begin{equation}
  \bm{\Psi}(\bm{q}, t) = D_1(t) \bm{\Psi}(\bm{q})
\end{equation}
for any displacement field $\bm{\Psi}(\bm{q})$ with the \textit{linear growth factor} $D_1(a)$ that satisfies,
\begin{equation}
  \label{eq:linear-growth-factor}
  \frac{d}{dt}\left( a^2 \dot{D}_1 \right)
  + \frac{3}{2} H_0^2 \Omega_{m,0} a^{-1} D_1 = 0.
\end{equation}
This is a second order linear differential equation; its general
solution is a linear combination of two special solutions. Usually,
one of them is a growing mode, and the other is a decaying mode. The
growing solution is known to be,
\begin{equation}
  D_1(a) \propto \tilde{D}(a) := H(a) \int_0^a \left( a'H(a') \right)^{-3} da',
\end{equation}
and the growth factor is normalised to be 1 at present as a convention,
\begin{equation}
  D(a) = \tilde{D}(a)/\tilde{D}(1)
\end{equation}
%
The \textit{linear growth rate} $f$ is the logatithmic derivative of $D_1$,
\begin{equation}
  f := \frac{d\ln D_1}{d\ln a}.
\end{equation}
Simple approximate solutions, $f \approx \Omega^{0.6}$ or
$\Omega^{0.55}$, are often used, but analytical derivative is also easy,
\begin{equation}
  f = \left[ (aH)^2 \tilde{D}(a) \right]^{-1} - \frac{3}{2} \Omega(a).
\end{equation}
%Peculiar velocity
%\begin{align}
%  \bm{v}_\mathrm{pec} &= a \dot{\bm{\Psi}}(\bm{q}, t)
%    = a (\dot{D}/D) \bm{\Psi}
%    = \frac{da}{dt} \frac{a}{D} \frac{dD}{da} \bm{\Psi}\\
%   &= a H\!f  \bm{\Psi}(\bm{q}, t)
%\end{align}

\subsection{2LPT}
\subsubsection{Position}
2nd order Lagrangian Perturabation Theory (2LPT)
\begin{equation}
  \bm{\Psi}(\bm{q}, a) =
    D_1(a) \bm{\Psi}^{(1)}(\bm{q}) + D_2(a) \bm{\Psi}^{(2)}(\bm{q}) + \cdots
\end{equation}
where, $\bm{\Psi}^{(1)}$ and $\bm{\Psi}^{(2)}$ are solutions of,
\begin{align}
  \bm{\nabla}_q \cdot \bm{\Psi}^{(1)} &= - \delta_0(\bm{q})\\
  \bm{\nabla}_q \cdot \bm{\Psi}^{(2)} &= \sum_{i < j}
    \left( \Psi^{(1)}_{i,i} \Psi^{(1)}_{j,j} - \Psi^{(1)}_{i,j} \Psi^{(1)}_{j,i}
    \right)
\end{align}
where $\delta_0(q)$ is a linear random Gaussian density contrast field
extrapolated to $a=1$ with the linear growth factor.
The equation for the 2nd-order growth rate is,
\begin{equation}
  \label{eq:second-growth-factor}
  \frac{d}{dt} ( a^2 \dot{D_2} )
    + \frac{3}{2}H_0^2 \Omega_{m,0} a^{-1} (-D_2 + D_1^2) = 0,
\end{equation}
whose approximate solution is known to be,
\begin{equation}
  D_2(a) = -\frac{3}{7} D_1(a)^2 \Omega(a)^{-1/143}
\end{equation}
%
In the simulation, 2LPT position is stored in $\texttt{dx1}$ and $\texttt{dx2}$,
\begin{equation}
  \bm{x}_\mathrm{2LPT} = \bm{x}_0 + D_1(a) d\bm{x}^{(1)} + D_2(a) d\bm{x}^{(2)}
\end{equation}
with,
\begin{align}
  &d\bm{x}^{(1)} \leftarrow \bm{\Psi}^{(1)},\\
  &d\bm{x}^{(2)} \leftarrow \bm{\Psi}^{(2)},\\
  &D_1(a) = \texttt{cosmology\_D\_growth}(a),\\
  &D_2(a) =  \texttt{cosmology\_D2\_growth}(a, D_1).          
\end{align}

\subsubsection{Velocity}
2LPT velocity can be obtained by time derivative,
%
\begin{equation}
  \bm{v}_\mathrm{2LPT} = a^2 \dot{\bm{x}}_\mathrm{2LPT} =
    a^2 \dot{D}_1 \bm{\Psi}^{(1)} + a^2 \dot{D}_2 \bm{\Psi}^{(2)},
\end{equation}
where,
\begin{align}
  \dot{D}_1 &= \frac{D_1}{a}\frac{da}{dt} = D_1(a) H(a) f,\\
  \dot{D}_2 &= 2 D_2(a) \frac{\dot{D}_1}{D_1} = 2 D_2(a) H(a) f.
\end{align}
Time derivaive of $\Omega^{-1/143}$ is neglected because it is very
small.\\


\noindent In the simulation, the 2LPT velocity is written as,
\begin{equation}
  \bm{v}_\mathrm{2LPT} = D_{1v}(a) d\bm{x}^{(1)} + D_{2v}(a) d\bm{x}^{(2)},
\end{equation}
with,
\begin{align}
  D_{1v}(a) &= a^2 \dot{D}_1(a) = \texttt{cosmology\_Dv\_growth}(a, D_1),\\
  D_{2v}(a) &= a^2 \dot{D}_2(a) = \texttt{cosmology\_D2v\_growth}(a, D_2)
\end{align}

\subsubsection{Acceleration}
The fist order is,
\begin{equation}
  \frac{d}{dt}\bm{v}^{(1)} = \frac{d}{dt} (a^2 \dot{D_1}) \Psi^{(1)}
  = -\frac{3}{2} \Omega_{m,0} a^{-1} D_1 \Psi^{(1)},
\end{equation}
where we use equation~(\ref{eq:linear-growth-factor}) for the linear growth factor. The second order with equation~(\ref{eq:second-growth-factor}) gives,
\begin{equation}
  \frac{d}{dt}\bm{v}^{(2)} = \frac{d}{dt} (a^2 \dot{D_2}) \Psi^{(2)}
  = -\frac{3}{2} H_0^2 \Omega_{m,0} a^{-1}
    \left(1 + \frac{3}{7} \Omega_m^{-1/143}\right) \Psi^{(2)}
\end{equation}
% ToDo: check sign of equation (31) and the above eq

\begin{equation}
  \bm{a}_\mathrm{2LPT} = -\frac{2}{3} \Omega_{m,0} a^{-1} \left[ D_{1}(a) d\bm{x}^{(1)} + D_{2a}(a) d\bm{x}^{(2)} \right],
\end{equation}
with,
\begin{equation}
  D_{2a}(a) = \texttt{cosmology\_D2a\_growth}(a, D_1)
\end{equation}


\end{document}
